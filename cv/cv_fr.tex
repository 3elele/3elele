%%%%% preamble %%%%%
%%%%% Change to report class if multiple files are used as chapters %%%%%
\documentclass[pdftex, a4paper, 11pt, twoside, french]{article}
%%%%% Change here the language spelling %%%%%
%%%%% inspired from https://www.overleaf.com/latex/templates/clean-cv-template/qqwnkbrspbtr

%%%%% Page-Layout
\pagestyle{empty}
\setlength\topmargin{-1.3in}
\setlength\oddsidemargin{-0.8in}
\setlength\textwidth{200mm}
\setlength\textheight{380mm}

%%%%% word wrapping with no truncation
\tolerance=1
\emergencystretch=\maxdimen
\hyphenpenalty=10000
\hbadness=10000

%%%%% Sans serif font use
\renewcommand{\familydefault}{\sfdefault}
%%%%% friendly text commands
\providecommand{\code}{\texttt}
\providecommand{\strong}{\textbf}
\providecommand{\italic}{\textit}

%%%%% Font Awesome custom usage
\newcommand{\faCenter}[1]{\raisebox{-2pt}{\faIcon{#1}}}

%%%%% Section formatting
\newcommand{\cvSection}[1]{
	{\color{white}\rule{\linewidth}{0.2pt}}
	\vspace{-0.5em}
	{\large \strong{#1}}
	{\color{teal}\rule{\linewidth}{1pt}}
}

\newcommand{\daterule}{\color{dateline}\vrule width 1pt}

%%%%% PACKAGES IMPORTS %%%%%%
\usepackage{babel}

%%%%% long error messages
\errorcontextlines=999

%%%%% better font, similar to the default springer font
%%%%% cfr-lm is preferred over lmodern. Reasoning at http://tex.stackexchange.com/a/247543/9075
\usepackage[
	rm={oldstyle=false,proportional=true},
	sf={oldstyle=false,proportional=true},
	tt={oldstyle=false,proportional=true,variable=true},
	qt=false% 
]{cfr-lm}

%%%%% extended enumerate, such as \begin{compactenum}
\usepackage{paralist}

%%%%% tweak \url{...}
\usepackage{url}
% nicer // - solution by http://tex.stackexchange.com/a/98470/9075
\makeatletter
\def\Url@twoslashes{\mathchar`\/\@ifnextchar/{\kern-.2em}{}}
\g@addto@macro\UrlSpecials{\do\/{\Url@twoslashes}}
\makeatother
\urlstyle{same}
% improve wrapping of URLs - hint by http://tex.stackexchange.com/a/10419/9075
\makeatletter
\g@addto@macro{\UrlBreaks}{\UrlOrds}
\makeatother

%%%%% enable hyperref without colors and without bookmarks
\usepackage[
	bookmarks=false,
	breaklinks=true,
	colorlinks=true,
	linkcolor=black,
	citecolor=black,
	urlcolor=black,
	pdfpagelayout=SinglePage,
	pdfstartview=Fit
]{hyperref}

%%%%% graphics
\usepackage[export]{adjustbox} % loads also graphicx
\usepackage{float}
\usepackage{rotating}
\usepackage{wrapfig}
\usepackage{fancyhdr}
\usepackage{ae}
\usepackage{color}

%%%%% IPA characters
\usepackage{tipa}

%%%%% Font Awesome Icons
\usepackage{fontawesome5}

%%%%% Font and table optimisations
\usepackage[utf8]{inputenc}
\usepackage[T1]{fontenc}
\usepackage[table,xcdraw]{xcolor}
\usepackage{longtable}
\usepackage{makecell}
\usepackage{multirow}

\definecolor{datecolor}{gray}{0.4}
\definecolor{dateline}{gray}{0.5}


\begin{document}
%%%%% Title
\begin{center}
	{\huge Gabriele CHIGNOLI, Dr} \\
	{\Large \texttt{\textsc{Tech Lead Python}}}
\end{center}
\vspace{-17.2pt}
{\color{teal}\rule{\linewidth}{1pt}}
\vspace{-1.5em}

\begin{center}
\begin{tabular}{llllll}
	\href{https://github.com/3elele}{\faCenter{github-square}} & \href{https://www.linkedin.com/in/gabrielechignoli/}{\faCenter{linkedin}} & \href{https://www.researchgate.net/profile/Gabriele-Chignoli}{\faCenter{researchgate}} & \faCenter{envelope} \href{mailto:gabrielechignoli@icloud.com}{gabrielechignoli@icloud.com}  & \faCenter{phone-alt} +33 620142541 & \faCenter{house-user} 75 rue du moulin vert 75014 Paris, France \\
\end{tabular}
\end{center}
\vspace{-1.5em}

\cvSection{Résumé}
\begin{itemize}
\vspace{-0.2em}
	\item[\faTasks] \textit{Gestion de projet} : supervision de projets de production de données au niveau national et européen ; gestion d'équipe pour assurer le partage des tâches et la réalisation des objectifs ; publication et maintien de bibliothèques logicielles publiques ou propriétaires.
\vspace{-0.5em}
	\item[\faLaptopCode] \textit{Programmation et langages de script} : Bash, C++(basique), Docker, Git, HTML, Java, Javascript (Vue.js), LateX, MatLab, Perl, Praat, Python, R, Swift, XML.
\vspace{-0.5em}
	\item[\faTable] \textit{Traitement de données} : capacité à gérer de grandes quantités de données dans des formats multiples formats ; traitement de grands corpora d'images, de signaux, de textes et de parole ;  conversion en données appropriées pour l'entraînement des réseaux neuronaux artificiels ; évaluation des réseaux neuronaux et de leurs réponses dans des tâches multiples.
\vspace{-0.5em}
	\item[\faCenter{spell-check}] \textit{Traitement Automatique du Langage Naturel} : formalisation de structures de pour l'analyse automatique avec applications multilingues ; ingénierie de requêtes et structuration de l'information pour l'évaluation et le développement de LLM.
\vspace{-0.5em}
	\item[\faCommentDots] \textit{Phonétique} : étude de l'influence de la position prosodique sur les voyelles, des différences entre l'écrit et l'oral, et de l'influence des caractéristiques individuelles des locuteurs sur les paramètres acoustiques ; annotation phonétique selon différentes conventions ; comparaison entre différentes techniques d'extraction de paramètres acoustiques à partir de corpus de parole.
\vspace{-0.5em}
	\item[\faFulcrum] \textit{Communication} : présentations effectuées dans plusieurs environnements (poster, conférence, enseignement, réunions d’équipe) ; animation de réunions autour de la résolution de problèmes et l'amélioration des méthodologies appliquées.
\vspace{-0.5em}
	\item[\faRing] \textit{Travail d’équipe} : capacité à gérer différents points de vue et priorités en raison de la participation à des projets impliquant plusieurs équipes de recherche ; pratique de sports d'équipe pendant 10 ans.
\vspace{-0.5em}
	\item[\faLanguage] \textit{Langues} : Italien et Napolitain  (langues maternelles) ; Anglais et Français (courant) ; Portugais (fluent) ; Espagnol (bonne compréhension) ; Allemand et Japonais (notions de base).
\end{itemize}
\vspace{-1.0em}

\cvSection{Expériences}
\begin{tabular}{>{\raggedleft\arraybackslash}p{8em}p{0.78\textwidth}}
	{\small \color{datecolor}04/2024 -- présent \daterule} & \textbf{Chercheur en IA} \textit{Advanced Topographic Development and Images, Paris}. \newline Recherche et développement ainsi que pilotage d'une équipe IA pour intégrer des technologies innovantes dans les solutions logicielles existantes : caractérisation et analyse de signaux provenant de sources connues ou inconnues ; traitement d'images pour l'amélioration de la topographie par la reconnaissance d'obstacles ; application de méthode de RAG pour un LLM spécifique au domaine des télécommunications. \newline \\
	{\small \color{datecolor}01/2023 -- 03/2024 \daterule} & \textbf{Chef de projet en technologies de la langue} \textit{Evaluations and Language resources Distribution Agency, Paris}. \newline Gestion et suivi de projets liés aux technologies de la langue pour des partenaires français et internationaux : Commission européenne (2 projets sur des langues sous-dotées et le partage des données linguistiques) ; Radio France (2 projets sur l'annotation des données) ; 3 autres projets pour des clients non divulgués sur l'amélioration des données. \newline \\
	{\small \color{datecolor}09/2018 -- 09/2022 \daterule} & \textbf{Doctorat en Phonétique, Phonologie et Sciences de la parole} \textit{Université Sorbonne Nouvelle - Centre National de la Recherche Scientifique - Laboratoire de Phonétique et Phonologie, Paris}. \newline Thèse en anglais “\href{https://hal.science/tel-03911819}{Speech components in phonetic characterisation of speakers: a study on complementarity and redundancy of conveyed information}”. Utilisation de techniques avancées d'analyse acoustique et d'apprentissage machine par réseaux de neurones convolutifs. \newline \\%au sein du projet VoxCrim, ANR-17-CE39-0016. \\
	{\small \color{datecolor}09/2017 -- 06/2021 \daterule} & \textbf{Enseignant}. \textit{Université Paris Descartes - Université Sorbonne Nouvelle, Paris}. \newline Cours en Informatique, Linguistique générale, Méthodologie de la recherche et Statistiques.
\end{tabular}

\cvSection{Formation}
\vspace{0.5em}
\begin{tabular}{>{\raggedleft\arraybackslash}p{8em}p{0.78\textwidth}}
	{\small \color{datecolor}09/2016 -- 07/2018 \daterule} & \textbf{Master en Traitement Automatique des Langues} \textit{Université Sorbonne Nouvelle, Paris}. \newline \textbf{Stage de fin d'études}. \textit{Laboratoire d'Informatique pour la Mécanique et les Sciences de l'Ingénieur, Orsay}. \\% \newline Dissertation in French "Reconnaissance Automatique du Locuteur, mécanisme humain et tâche informatique : application de méthodes statistiques". \\
	{\small \color{datecolor}09/2014 -- 05/2016 \daterule} & \textbf{Masters en Sociolinguistique et en Relations internationales}. \newline \textit{Université Sorbonne Nouvelle, Paris}. \\
	{\small \color{datecolor}10/2011 -- 07/2014 \daterule} &  \textbf{Licence, LPP Erasmus en Linguistique générale et comparée} \textit{Università Suor Orsola Benincasa, Naples - Université Sorbonne, Paris}. \\%\newline Mémoire en italien "L'Europa cambia l'Europa". Prix académique Paolo Iannotti pour diplômés excellents. \\
\end{tabular}
\end{document}